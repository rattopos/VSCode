\documentclass[a4paper,12pt]{article}

\usepackage{amssymb,amsthm}
\usepackage{mathtools}
\usepackage{kotex}
\usepackage{enumitem}
\setmainhangulfont{UnBatang.ttf}[
	BoldFont={*Bold},
	ItalicFont={*},
	ItalicFeatures={FakeSlant={.16}},
	BoldItalicFont={*Bold},
	BoldItalicFeatures={FakeSlant={.16}},
	SmallCapsFont={UnGraphic.ttf},
	SmallCapsFeatures={Scale=MatchLowercase}
]

\theoremstyle{definition}
\newtheorem{definition}{정의}
\newtheorem{example}{예시}
\newtheorem{exercise}{연습}

\theoremstyle{plain}
\newtheorem{theorem}{정리}
\newtheorem{lemma}{보조정리}
\newtheorem{proposition}{명제}
\newtheorem{corollary}{따름정리}

\theoremstyle{remark}
\newtheorem*{remark}{주목}
\newtheorem*{claim}{주장}


\title{양자 현상의 논리: Kochen-Specker 정리}
\author{이창재}

\begin{document}
\maketitle

\section{고전적인 해석 시도}

양자역학에서 정규헬륨(orthohelium)의 스핀은 다음 성질들은 측정가능하다.

\begin{enumerate}[label=(\alph*)]
    \item \(t\) 순간 스핀의 \(\alpha \in S^2\) 방향 사영 \(s(\alpha,t)\)
    \item 직교기준틀 \(\{\alpha_1,\alpha_2,\alpha_3\} \subset S^2\)가 주어졌을때 \(t\) 순간의 사영의 길이 \(|s|(\alpha_i,t)\)
    \item \(s(\alpha,t)\)는 값을 오직 -1,0,1을 갖는 확률변수이다.
    \item 어느 순간 \(t\)에서든 \(\sum_{i=1}^{3} |s|(\alpha_i,t)=2\)
\end{enumerate}

\paragraph*{가설 A} \(\Omega\)라는 내부상태들의 공간이 존재하고 각 내부상태 \(\omega \in \Omega\)에 의해 시간 \(t\)에서 스핀의 \(\alpha\)-축 사영의 참값 \(s(\alpha,t;\omega)\)가 정해진다.

\paragraph*{가설 B} (c)의 확률적인 측면은 \(\omega=\omega(t)\)의 정확한 값을 몰라서 발생한다. 따라서,

\[(s(\alpha,t)\mbox{의 수학적 기댓값}) = \int_{\Omega} s(\alpha,t;\omega)\, d\mu(\omega)\]

\begin{proposition}
    임의의 기준틀 \(\{\alpha_1,\alpha_2,\alpha_3\}\) 마다 정확히 한 축에 대해서 영값을 가지는 사상 \(S^2 \to \{0,1\}\)은 존재하지 않는다. 나아가, 다음 성질을 가진 117개의 점으로 구성된 유한계를 만들 수 있다. 
\end{proposition}

\section{비상대론적 양자역학의 언어}

\end{document}