\documentclass{beamer}
\usepackage{mathtools}
\usepackage{amssymb}
\usetheme{Madrid}
\usecolortheme{default}

\title{Gleason's Theorem and Quantum Logic}
\author{Changjae Lee\inst{1}}
\institute[SNU]{
    \inst{1}
    Department of Mathematical Science\\
    Seoul National University
}
\date{November 29, 2024}

\begin{document}

\frame{\titlepage}

\AtBeginSection[]{
    \begin{frame}
        \frametitle{Table of Contents}
        \tableofcontents[currentsection]
    \end{frame}
}
\begin{frame}
    \frametitle{Table of Contents}
    \tableofcontents
\end{frame}

\section{Introduction}
\begin{frame}
    \frametitle{Introduction}
Quantum logic began in 1936 when John von Neumann and Garret Birkhoff described the quantum propositions, through a lattice of projection operators in Hilbert space.\pause This was a new logic that, unlike the Boolean logic, did not satisfy the distribution law.\pause

\begin{examples}
    Choose a metric system \(\hbar=1\),\pause and let \\
    P=``the particle has momentum in the interval \([0,\frac{1}{6}]\)",\pause\\
    Q=``the particle is in the interval \([-1,1]\)",\pause\\
    R=``the particle is in the interval \([1,3]\)",\pause then
    \begin{gather*}
        P \wedge (Q\vee R) = 1,\\
    (P \wedge Q) \vee (P \wedge R) = 0.
    \end{gather*}
    by the uncertainty principle
    \(\sigma_{x} \sigma_{p} \ge \hbar/2=\frac{1}{2}\).
\end{examples}

\end{frame}


\section{Operator Algebras}

\begin{frame}{Operator Algebras}
    \begin{itemize}
        \item \textbf{Banach space} \(\mathcal{B}=(B,\left \lVert \cdot \right \rVert)\) is a normed vector space which is complete with respect to induced distance \(d(x,y)=\left \lVert x-y \right \rVert\). \pause
        \item \textbf{Hilbert space} \(\mathcal{H}=(H,\left \langle \cdot,\cdot \right \rangle)\) is a inner product space which is complete with respect to induced norm \(\left \lVert x \right \rVert =\sqrt{\left \langle x,x \right \rangle}\).\pause
    \end{itemize}
    A Hilbert space \(\mathcal{H}\) is \textbf{separable} if it has countable dense subset.
    \begin{block}{Theorem}
        Hilbert space is separable if and only if it has orthonormal basis.
    \end{block}
\end{frame}

\begin{frame}
    \frametitle{Operator Algebras}
        \textbf{Banach algebra} is an associative algebra \(A\) over the real or complex that at the same time is also Banach space which satisfies
        \[\left \lVert xy \right \rVert \le \left \lVert x \right \rVert \left \lVert y \right \rVert\]\pause
        
        \textbf{\(C^\ast\)-algebra} is a complex Banach algebra \(A\) together with an involution \(\ast:A \to A, x \mapsto x^\ast\) which satisfies\pause
            \begin{enumerate}
                \item \(x^{\ast\ast}=x\),\pause
                \item \((x+y)^\ast=x^\ast+y^\ast\),\pause
                \item \((xy)^\ast = y^\ast x^\ast\),\pause
                \item \((\lambda x)^\ast=\bar{\lambda}x^\ast\) for all \(\lambda \in \mathbb{C}\),\pause
                \item \(\left \lVert xx^\ast \right \rVert = \left \lVert x \right \rVert \left \lVert x^\ast \right \rVert \)
            \end{enumerate}\pause
        \begin{example}
            \(\mathcal{B}(H)\): The set of all bounded linear operators on a complex Hilbert space \(H\) is a \(C^\ast\)-algebra.
        \end{example}
    \end{frame}

        \begin{frame}{Operator Algebras}
            An element \(A\) of a \(C^\ast\)-algebra \(\mathcal{A}\) is said to be
            \begin{itemize}
                \item \textbf{normal} if \(AA^\ast=A^\ast A\)
                \item \textbf{self-adjoint} if \(A=A^\ast\)
                \item \textbf{projection} or \textbf{orthoprojector} if self-adjoint and \(A^2 = A\)
                \item \textbf{positive} if \(A=B^\ast B\) for some \(B \in \mathcal{A}\), and we write \(A \ge 0\).
            \end{itemize}
            Let \(\mathcal{A}\) and \(\mathcal{B}\) be two \(C^\ast\)-algebras; a mapping linear map \(\pi:\mathcal{A} \to \mathcal{B}\) such that
            \begin{enumerate}
                \item \(\pi(AB)=\pi(A)\pi(B)\),
                \item \(\pi(A^\ast)=\pi(A)^\ast\)
            \end{enumerate}
            is said to be a \textbf{\(\ast\)-morphism}. A \textbf{representation} of a \(C^\ast\)-algebra \(\mathcal{A}\) is defined to be a pair \((H,\pi)\), where \(H\) is a complex Hilbert space and \(\pi\) is a \(\ast\)-morphism from \(\mathcal{A}\) into \(\mathcal{B}(H)\).
        \end{frame}
        \begin{frame}{Operator Algebras}
            A \textbf{cyclic representation} of a \(C^\ast\)-algebra \(\mathcal{A}\) is defined to be a triple \((H,\pi,x)\) where \((H,\pi)\) is a representation of \(\mathcal{A}\) and \(x \in H\) is a cyclic vector for \(\{\pi(A): A \in \mathcal{A}\}\).
            \begin{block}{GNS representation}
                Let \(\omega\) be a state over the \(C^\ast\)-algebra \(\mathcal{A}\). then there exists a cyclic rperesentation \((H_{\omega},\pi_{\omega},x_{\omega})\) of \(\mathcal{A}\) such that
                \[\omega(A) = \left \langle x,\pi(A)x_{\omega} \right \rangle\]
            \end{block}
            for all \(A \in \mathcal{A}\), where \(\left \lVert x_{\omega} \right \rVert=\left \lVert \omega \right \rVert=1\)
        \end{frame}

        \begin{frame}{Operator Algebras}
            Let \(H\) be a Hilbert space. For any \(\mathcal{M} \subset \mathcal{B}(H)\), let \[\mathcal{M}'=\{A \in \mathcal{M} \mid AM=MA \mbox{ for all } M \in \mathcal{M}\}\]
            
        A \textbf{von Neumann algebra} on a Hilbert space \(H\) is a subset \(\mathcal{A}\) of \(\mathcal{B}(H)\) such that
        \begin{enumerate}
            \item \(I \in \mathcal{A}\),
            \item \(\alpha A + \beta B \in \mathcal{A}\) whenever \(A,b \in \mathcal{A}\), \(\alpha,\beta \in \mathbb{C}\),
            \item if \(A \in \mathcal{A}\), then \(A^\ast \in \mathcal{A}\),
            \item \(\mathcal{A}\) is closed in the weak operator topology.
        \end{enumerate}
        \end{frame}

        \section{Dirac-von Neumann Axioms}
        \begin{frame}
            \frametitle{Dirac-von Neumann Axioms}
            
            The Dirac-von Neumann axioms give a mathematical formulation of quantum mechanics in terms of operators on a Hilbert space. They were introduced by Paul Dirac in 1930 and John von Neumann in 1932.\pause
            
            \begin{block}{Hilbert Space Formulation}
            Let \(\mathcal{H}\) be a fixed complex Hilbert space of countably infinite dimension.\pause
            \begin{itemize}
                \item The observables of a qunatum system are defined to be the self-adjoint operators \(A\) on \(\mathcal{H}\).\pause
                \item A state \(\psi\) of the quantum system is a unit vector of \(\mathcal{H}\), up to scalar multiples.\pause
                \item The expectation value of an observable \(A\) for a system in a state \(\psi\) is given by the inner product \(\left \langle \psi,A\psi \right \rangle\).
            \end{itemize}
        \end{block}
    \end{frame}
    \begin{frame}{Dirac-von Neumann Axioms}
        \begin{block}{Operator Algebra Formalism}
            The Dirac-Von Neumann axioms can be formulated in terms of a \(C^\ast\)-algebra as follows.\pause
            \begin{itemize}
                \item The \textbf{bounded observables} of the quantum mechanical system are defined to be the self-adjoint elements of the \(C^\ast\)-algebra.\pause
                \item The \textbf{states} of the quantum mechanical system are defined to be positive functional \(\omega\) such that \(\left \lVert \omega \right \rVert=1\).\pause
                \item The value \(\omega(A)\) of a state \(\omega\) on an element \(A\) is the expectation value of the observable \(A\) if the quantum system is in the state \(\rho\).\pause
            \end{itemize}
        \end{block}        
        \begin{example}
            For \(C^\ast\)-algebra \(\mathcal{A}\) on Hilbert space \(\mathcal{H}\),
            \(\omega_x(A)=\left \langle x,Ax \right \rangle\) for \(x \in \mathcal{H}, A \in \mathcal{A}\) is called \textbf{vector state}.
        \end{example}
    \end{frame}
    
\section{Quantum Logic}

\begin{frame}
    \frametitle{Quantum Logic}
    A \textbf{lattice} \(\mathcal{L}=(L,\vee,\wedge)\) is a structure over a poset \((L,\le)\) such that
    \begin{gather*}
        x\wedge y = \inf \{x,y\}, \\
        x \vee y = \sup \{x,y\}.
    \end{gather*}
    exist in \(L\) for all \(x,y \in L\).\pause More generally, \emph{\(\sigma\)-lattice} or a \textbf{complete lattice} whenever
    \begin{gather*}
        \bigwedge_{i\in I} x_i = \inf \{x_i \mid i \in I\}, \\
        \bigvee_{i\in I} x_i = \sup \{x_i \mid i \in I\}.
    \end{gather*}
    exist in \(L\) for any sequence \((x_i)_{i \in I}\) with a countable or arbitary index set \(I\).\pause A \textbf{bounded lattice} \(\mathcal{L}=(L,\vee,\wedge,0,1)\) is a lattice which satisfy
    \[ 0 \le x \le 1 \quad (\forall x \in L) \]
\end{frame}

\begin{frame}
    \frametitle{Quantum Logic}
We say that a lattice \(\mathcal{L}\) is \textbf{distributive} if for all \(a,b,c \in L\),
\begin{gather*}
    a\wedge(b\vee c) = (a\wedge b) \vee (a\wedge c),\\
    a \vee (b \wedge c) = (a \vee b) \wedge (a\vee c).
\end{gather*}\pause
A lattice \(\mathcal{L}\) is modular if
\[a \vee ( b \wedge c) = (a\vee b)\wedge c \quad (a\le c, a,b,c \in L)\]
We say that for a unary operation \(': L\to L\) the \textbf{de Morgan laws} holds if for any sequence \((a_i)\) of elements from L we have
\begin{gather*}
    (\bigwedge_i a_i)' = \bigvee_i a'_i, \\
    (\bigvee_i a_i)' = \bigwedge_i a'_i
\end{gather*}

\end{frame}

\begin{frame}
    \frametitle{Quantum Logic}
We say \(\perp: L \to L, a \mapsto a^\perp \) is said to be an \textbf{orthocomplementation} on a poset \(L\) with 0 and 1 if
\begin{enumerate}
    \item \((a^\perp)^\perp = a\) for any \(a \in L\),\pause\\
    \item if \(a \le b\), then \(b^\perp \le a^\perp \),\pause
    \item \(a \vee a^\perp=1\) for any \(a\in L\).\pause
\end{enumerate}
The poset \(L\) with the orthocomplementation \(\perp\) is said to be \textbf{orthocomplemented}.\pause Two elements \(a\) and \(b\) are said to be \textbf{orthogonal}, and we write \(a\perp b\), if \(a\le b^\perp\).


\end{frame}

\begin{frame}
    \frametitle{Quantum Logic}
    An orthocomplemented poset \(L\) is said to be an \textbf{orthomodular poset} (in short OMP) if \pause
\begin{enumerate}
    \item \(a\vee b \in L\) whenever \(a\perp b,a,b \in L\),\pause
    \item (orthomodularity) if \(a\le b\), then \(b=a\vee(b\wedge a^\perp)\).\pause
\end{enumerate}
    For OMP \(L\), if we have
    \[\bigvee^{\infty}_{i=1} a_i \in L\]
    whenever \(a_i \perp a_j\) for \(i \neq j\), \(\{a_i\} \subset L\), then \(\mathcal{L}=(L,\vee,\wedge,0,1,\perp)\) is said to be a \textbf{quantum logic}.
\end{frame}

\begin{frame}{Quantum Logic}
    \begin{itemize}
        \item A \textbf{(finitely additive) measure} \(\mu\) on an OMP \(L\) is a map \(\mu: L \to \mathbb{C}\) such that \(\mu(x\vee y)=\mu(x) + \mu(y)\) whenever \(x \perp y\).\pause
        \item A measure \(\mu\) on OMP \(L\) is called a \textbf{state} if \(\mu\) has values in the unit interval \([0,1]\) and \(\mu(1)=1\).\pause
        \item A measure \(\mu\) is called \textbf{\(\sigma\)-additive} if \[\mu(\bigvee_{i=1}^{\infty}a_i)=\sum_{i=1}^{\infty} \mu(a_i)\] for any sequence \(a_i\) for \(a_i\perp a_j\) for \(i \neq j\).\pause
        \item A measure \(\mu\) is called \textbf{completely additive} if \[\mu(\bigvee_{i \in I} a_i)=\sum_{i\in I} \mu(a_i)\] for any index set \(I\) and for any sequence \(a_i\) for \(a_i\perp a_j\) for \(i \neq j\).
    \end{itemize}
    
\end{frame}

\begin{frame}{Quantum Logic}
    \begin{example}
        Let \(H\) be a real or complex Hilbert space. Denote by \(L(H)\) the system of all closed subspaces of \(H\). Then \(L(H)\) is a quantum logic (called the quantum logic of a Hilbert space \(H\)), where the partial ordering is determined by the set-theoretic inclusion, the meet and join are defined as follows
        \[\bigvee_t M_t = \bigcap_t M_t, \quad \bigvee_t M_t = cl(sp(\bigcup_t M_t)),\]
        and the orthocomplementation \[\perp: M \mapsto M^\perp = \{x \in H \mid \left \langle x,y \right \rangle =0 \mbox{ for all } y \in M\}.\]
    \end{example}
\end{frame}

\begin{frame}{Quantum logic}
    \begin{example}
        Suppose that \(\mathcal{A}\) is a von Neumann algebra of operators action on a real or complex Hilbert space \(H\). Denote by \(L_{\mathcal{A}(H)}\) the set of all closed subspaces of \(H\) whose orthoprojectors belongs to \(\mathcal{A}\). Then \(L_{\mathcal{A}(H)}\) is a sublogic of \(L(H)\).
    \end{example}\pause
    \begin{example}
        Let \(\mathcal{P}(\mathcal{A})\) denote the set of all projections in a \(C^\ast\)-algebra \(\mathcal{A}\). Projections \(P_1\) and \(P_2\) are called orthogonal if \(P_1P_2 =0\). Give an ordering on \(\mathcal{P}(\mathcal{A})\) by \(P_1 \le P_2\) if and only if \(P_1P_2=P_2P_1=P_1\). If \(\mathcal{A}\) is unital, then the structure \(\mathcal{P}(\mathcal{A})\) is an OMP with the complement \(P^perp=1-P\). The structure \((\mathcal{P}(\mathcal{A}),\le,0,1,\perp)\) is a Boolean algebra if and only if, \(A\) is abelian.
    \end{example}
\end{frame}

\section{Gleason's Theorem}
\begin{frame}
    \frametitle{Gleason's Theorem}
\begin{block}{Gleason's Theorem}
    If \(H\) be a separable, real or complex Hilbert space, \(\dim H \neq 2\), then for any state \(\mu\) on \(L(H)\) there exists a unique positive Hermitian trace operator \(T\) on \(H\) with \(\mathrm{tr} T =1\), such that
    \[\mu(M) = \mathrm{tr}(TP_M)\]
    for all \(M \in L(H)\).
\end{block}
proof. \cite[p.131-149]{MR1256736}
\end{frame}

\begin{frame}{Gleason's Theorem}
\begin{block}{Remark}
    Gleason's theorem can be reformulated into an equivalent form: For any state \(\mu\) on \(L(H)\) of a separable Hilbert space \(H\), \(\dim H \neq 2\), there exists an orthornormal system of vectors \((x_i)\) and a system of positive numbers \(\{\lambda_i\}\) such that \(\sum_i \lambda_i=1\), and \[\mu(M) = \sum_i \lambda_i \mu_{x_i}\]
    for all \(M \in L(H)\).
\end{block}
\end{frame}

\section{Hidden Variable Problem}
\begin{frame}{Hidden Variable Problem}
    \begin{block}{Theorem} Let \(H\) be a Hilbert space with \(\dim \mathcal{H}\ge 3\). There is no 0-1 finitely additive state on the projection lattice \(\mathcal{P}(H)\).
    \end{block}\pause
    proof. \cite[Theorem 3.4.1]{MR2015280}
    The nonexistence of 0-1 state on an OML is connected with a long dispute on hidden variables in quantum theory.\pause The existence of sufficiently many hidden variables would mean that the probabilistic character of quantum mechnics disappears by adding certain new parameters.\pause The theorem says theat the projection lattice of a Hilbert space cannot be embedded into a Boolean algebra because any such embedding would induce a 0-1 state.\pause It is therefore an important consequence of Gleason Theorem that the Hilbert space model cannot be completed by considering axiliary hidden variables to a classical model without violating its inner structure.

   
\end{frame}
\begin{frame}{Hidden Variable Problem}
 Gleason's theorem says that the probability structure is determined by quantum logic, i.e. by the ordered structure of projections in a Hilbert space.\pause The fact that only linear restriction can qualify for being quantum state has, besides hidden variable theory, other interesting physical consequences.\pause One of them is indeterminacy principle. \cite[p.87-93]{MR2015280}
 \begin{block}{Indeterminancy Priciple}
    Let \(p\) and \(q\) be atomic nonorthogonal projections in a Hilbert space \(H\). For every completely additive state \(\mu\) on \(\mathcal{P}(H)\) the following equivalence holds
    \[\mu(p),\,\mu(q) \in \{0,1\} \Leftrightarrow \mu(p)=\mu(q)=0\]
    In other words, no quantum state assigns sharp probability zero or one to two atomic nonorthogonal projection unless they are both false.
 \end{block}   
    
\end{frame}
\section{Quantum Probability}
\begin{frame}{Quantum Probability}
    \cite[p.8]{redei2006quantumprobabilitytheory} In classical probability theory, the probability space is a triple \((X,\Sigma,\mu)\) where \(X\) is a nonempty set, \(\Sigma\) is a \(\sigma\)-algebra on \(X\), and probability measure \(\Sigma \to [0,1]\).\pause
    
    In \textbf{noncommutative probability theory} or \textbf{qunatum probability theory}, a probability space is a triple \((\mathcal{M},P(\mathcal{M}),\phi)\) consisting of a von Neumann algebra, its lattice of orthogonal projections and a normal state on the algebra.\pause If \(\mathcal{M}\) is abelian, then \((\mathcal{M},P(\mathcal{M}),\phi)\) is just classical probability space.\pause
    
    A state \(\phi\) on a von Neumann algebra \(\mathcal{M}\) is said to be \textbf{disperision free} if \(\phi(A^2)-\phi(A)^2=0\), for all self-adjoint \(A \in \mathcal{M}\). A nonabelian factor admits no dispersion free state. That is one of characteristic of noncommutative probability.
\end{frame}
\begin{frame}{bibliography}
    \bibliographystyle{alpha}
    \bibliography{refs}
\end{frame}

\begin{frame}
\begin{center}
    \Huge{Thank You!}
\end{center}
\end{frame}

\end{document}