\documentclass[a4paper,11pt]{report}
\usepackage{mathtools}
\usepackage{amssymb}
\usepackage{amsthm}
\theoremstyle{definition}
\newtheorem{definition}{Definition}
\newtheorem{example}{Example}
\newtheorem{exercise}{Exercise}

\theoremstyle{plain}
\newtheorem{theorem}{Theorem}
\newtheorem{lemma}{Lemma}
\newtheorem{proposition}{Proposition}
\newtheorem{corollary}{Corollary}

\theoremstyle{remark}
\newtheorem*{remark}{Remark}
\newtheorem*{claim}{Claim}

\begin{document}
\chapter*{Introduction}

In 1932, John von Neumann noticed that the projection operators of Hillbert space could be viewed as quantum mechanical propositions about observables \cite{Neumann2018}. The principles governing these quantum propositions were then called quantum logic by von Neumann and Birkhoff \cite{MR1503312}. Von Neumann also managed to dervie the Born rule in his textbook \cite{Neumann2018}. However, his assumptions were regarded to not be well-motivated by John Bell \cite{MR208927}, \cite{MR2726358}.

By the late 1940s, George Mackey wondered whether the Born rule was the only possible rule for calculating probabilities in a theory that represented measurements as orthonormal bases on a Hilbert space. In his investigation of mathematical foundation of quantum mechanics, Mackey had proposed the following problem \cite[p.50-51]{MR96112}\cite[p.129]{MR1256736}: Determine all measures on the closed subspaces of a Hilbert space. Gleason provided the answer to this problem in the same year \cite{MR96113}, which is later called Gleason's theorem. Gleason succeeded in decribing all \(\sigma\)-additive probability measures on the logical of all closed subspaces of a separable Hilbert space and in showing that, except for the ovious two-dimensional counterexamplaes, all probability measures can be identified with normal states in the sense of von Neumann approach. Gleason's achievement confirmed von Neumann's original insight and put the calculus of Hilbert space quantum mechanics on natural physical grounds \cite[p.87-88]{MR2015280}. Gleason's theorem is of particular importance for the field of quantum logic and its attempt to find a minimal set of mathematical axioms for quantum theory.

After a considerable effort the Gleason's theorem was established in the early 90's for finitely additive vector measures on the projection lattices of von Neumann algebras. It has turned out that the lattice homomorphisms on nonabelian von Neumann algebras are \(\sigma\)-additve, or that finitely additive measures on projection lattices whose kernels are lattice ideal enjoy many continuity properties \cite[p.3-4]{MR2015280}.

The present paper is intended to serve as an introduction to topic of the relation between Gleason's theorem and the theory of quantum logic. Most of the contents were referred to chapters 1, 2, 3 and 5 of \cite{MR1256736} and chapters 1, 2, 3, and 9 of \cite{MR2015280}.


\chapter{Preliminaries}

\section{Elements of Operator Algebras}

A \(C^\ast\)-algebra \(A\) is a Banach algebra over the field \(\mathbb{C}\) together with a map \(\ast: A \to A\), \(a \mapsto a^\ast\) which following properties:
\begin{enumerate}
    \item \(a^{\ast\ast}=a\) for all \(a \in A\),
    \item \((a+b)^\ast=a^\ast+b^\ast\) for all \(a,b \in A\),
    \item \((ab)^\ast = b^\ast a^\ast\) for all \(a,b \in A\),
    \item \((\lambda a)^\ast = \bar{\lambda}a^\ast\) for all \(\lambda \in \mathbb{C}\) and \(a \in A\),
    \item \(\left \lVert aa^\ast \right \rVert = \left \lVert a \right \rVert \left \lVert a^\ast \right \rVert\) for all \(a \in A\).
\end{enumerate}

A bounded linear map \(\pi:A \to B\) between \(C^\ast\)-algebras \(A\) and \(B\) is called a \(\ast\)-homomorphism if
\begin{enumerate}
    \item \(\pi(ab)=\pi(a)\pi(b)\)
    \item \(\pi(a^\ast) = \pi(a)^\ast\)
\end{enumerate}
for all \(a,b \in A\).

Let \(A\) be a \(C^\ast\)-algebra. A \emph{state} in \(C^\ast\)-algebra is a positive functional \(\rho\) which is \(\left \lVert \rho \right \rVert=1\). We shall denote the set of all states on \(A\) by \(S(A)\) which is called the \emph{state space} of A.

A \(\ast\)-representation of a \(C^\ast\)-algebra \(A\) on a Hilbert space \(H\) is a ring homomorphism \(\pi: A \to B(H)=\{f \mid f \mbox{ is a bounded linear operator on }H\}\) such that
\begin{enumerate}
    \item \(\pi(a^\ast)=\pi(a)^\ast\) for all \(a \in A\),
    \item \(\pi\) is nondegenerate.
\end{enumerate}

\begin{theorem}[GNS representation]
    Given a state \(\rho\) of \(A\), there is a \(\ast\)-representation \(\pi\) of \(A\) acting on a Hilbert space \(H\) with unit cylic vector \(\psi\) such that
    \[\rho(a)=\left \langle \pi(a) \psi,\psi  \right \rangle\]
    for every \(a\) in \(A\). Moreover, the representation is unique up to unitary equivalence.
\end{theorem}
\begin{proof}
    \cite[Theorem 4.5.2]{MR1468229}
\end{proof}

Let \(X\) be an inner product space. A \emph{spilitting subspace} of \(X\) is the subspace \(M\) of \(X\) which is the following property holds,
\[M+M^\perp=S.\]
Then any vector \(x \in X\) can be uniquely expressed in the form
\[x=x_M+x_{M^{\perp}},\]
where \(x_M \in M\) and \(x_{M^\perp} \in M^\perp\). We denote by \(E(X)\) the set of all splitting subspaces.
The map \(P_M:X\to X\) such that \(P_M(x)=x_M\) for all \(x \in X\) is a bounded linear operator with \(\left \lVert P_M \right \rVert=1\) whenever \(M \neq \{0\}\). Moreover, \(P^2_M=P_M\), \(P_M\) is self-adjoint, and \(\mathrm{ran}\, P_M = M\). The operator \(P_M\) is called the \emph{orthoprojector} or \emph{projection} from \(X\) onto \(M\). It is clear that \(I-P_M\) is an orthoprojector onto \(M^\perp\).
\begin{proposition}
    Let \(P\) is an idempotent linear operator on \(X\) which is self-adjoint. If \(M=\{x\in X \mid Px=x\}\), then \(M \in E(S)\) and \(P_M = P\).
    \begin{proof}
        We have
        \[\left \langle x,z-Pz \right \rangle=\left \langle x,z \right \rangle-\left \langle x,Pz \right \rangle=\left \langle x-Px,z \right \rangle=0\]
        for all \(x \in M\) and for all \(z \in X\). Therefore, \(z-Pz \in M^\perp\), and \[z=Pz+(I-P)z\]
    \end{proof}
\end{proposition}
\begin{corollary}
    If \(P\) is the orthoprojector onto a splitting subspaces \(M\) of \(X\), then \(I_P\) is the orth projector onto \(M^\perp\) and \(M^\perp=\{x\in X \mid Px=0\}\).
\end{corollary}
\begin{proposition}
    Let \(\{e_1,\dots,e_n\}\) is a finite orthonormal system in \(X\). Then \(M=\{\mathrm{sp}(e_1,\dots,e_n) \in E(S)\}\) and
    \[P_M(x) = \sum_{i=1}^{n} (x,e_i)e_i\]
    for all \(x \in X\).
\end{proposition}
\begin{lemma}
A splitting subspace \(M\) of \(X\) is invariant under an operator \(T\) on \(X\) if and only if \(AP_M=P_MTP_M\).
\end{lemma}
\begin{lemma}
Let \(M\) and \(N\) be two splitting subspaces. Then the followings are equivalent.
\begin{enumerate}
    \item \(M \perp N\).
    \item \(P_M P_N = O\)
\end{enumerate}
\begin{proof}
    If \(M\perp N\), then \(N \subset M^\perp\). We have \(P_Nx \in N\) for all \(x \in X\), so that \(P_MP_N=0\) for all \(x \in X\). If \(P_M P_N=O\), conversely, then \(P_Mx=P_M P_N x =0\) for all \(x \in N\). Therefore \(N \subset M^\perp\), so that \(M\perp N\).
\end{proof}
\end{lemma}
We say that two orthoporjectors \(P\) and \(Q\) on \(X\) are \emph{orthogonal} if \(PQ=O\), and we write \(P\perp Q\).
\begin{lemma}
    Let \(M\) and \(N\) be two splitting subspaces of \(X\). The following statements are equivalent:
    \begin{enumerate}
        \item \(P_M \le P_N\).
        \item \(\left \lVert P_M \right \rVert \le \left \lVert P_Nx \right \rVert\) for all \(x \in X\).
        \item \(M \subset N\).
        \item \(P_M P_N = P_M\).
        \item \(P_N P_M = P_M\).
    \end{enumerate}
\end{lemma}

\section{Quantum Logic}

Let \(L\) be a poset. For a sequence \((a_i)_{i\in I}\) in \(L\), the \emph{join} is defined by \[\bigvee_{i \in I} a_i = \sup \{a_i \mid i \in I\},\]
and the \emph{meet} is defined by
    \[\bigwedge_{i \in I} a_i = \inf \{a_i \mid i \in I\}.\]
    \begin{definition}
        Let \(L\) be a poset. Then
        \begin{itemize}
            \item \(L\) is a \emph{lattice} if \(a\vee b\) and \(a \wedge b\) exist in L for any \(a,b \in L\).
            \item \(L\) is a \emph{\(\sigma\)-lattice} if \(\bigvee_{i \in I} a_i\) and \(\bigwedge_{i\in I} a_i\) exist in \(L\) for any sequence \((a_i)_{i\in I}\) with a countable index set \(I\).
            \item \(L\) is a \emph{complete lattice} if \(\bigvee_{i \in I} a_i\) and \(\bigwedge_{i \in I} a_i\) exist in \(L\) for any sequence \((a_i)_{i\in I}\) with an arbitrary index set \(I\).
        \end{itemize}
    \end{definition}
\begin{definition}
    A poset \(L\) is said to be \emph{bounded} if there exist elements \(0\) and \(1\) in \(L\),
    \[0\le x \le 1\]
    for all \(x \in L\).
\end{definition}
A unary operation \(\perp: L \to L, a \mapsto a^\perp\) is said to be an \emph{orthocomplementation} on a bounded poset \(L\) if
\begin{enumerate}
    \item \((a^\perp)^\perp=a\),
    \item if \(a \le b\), then \(b^\perp\le a^\perp\),
    \item \(a\vee a^\perp=1\)
\end{enumerate}
for any \(a,b \in L\). A bounded poset \(L\) with the orthocomplementation \(\perp\) is called \emph{orthocomplemented}.
\begin{definition}
    An orthocomplemented poset \(L\) is said to be an \emph{orthomodular poset} (in short OMP) if
    \begin{enumerate}
        \item \(a \vee b \in L\) if \(a\perp b\),
        \item (orthomodularity) \(b=a\vee(b \wedge a^\perp)\) if \(a \le b\),
    \end{enumerate}
    hold for all \(a,b \in L\).
\end{definition}
We denote an orthomodular lattice by OML. If an OMP \(L\) have the following property
\[\bigvee_{i=1}^\infty a_i \in L\]
for any pairwise orthgonal sequence \((a_i)_{i=1}^{\infty}\),
\(L\) is called a \emph{quantum logic}.
\begin{definition}
    Let \(L_1\) amd \(L_2\) be two OMPs. A mapping \(h:L_1 \to L_2\) is called a \emph{homomorphism} if
    \begin{enumerate}
        \item \(h(1)=1\),
        \item \(h(a)\perp h(b)\) if \(a\perp b\) in \(L_1\),
        \item \(h(a\vee b)=h(a)\vee h(b)\) if \(a\perp b\) in \(L_1\).
    \end{enumerate}
    In addition, we call \(h\) a \(\sigma\)-homomorphism if
    \[h(\bigvee_{i=1}^{\infty} a_i) = \bigvee_{i=1}^{\infty} h(a_i)\]
    holds for any sequence of pairwise orthogonal elements \((a_i)_{i=1}^{\infty}\) of \(L_1\).
\end{definition}
\begin{definition}
    Let \(L\) be an OMP. Consider a map \(\mu:L\to [-\infty,\infty]\) such that
    \begin{gather}
        \mu(0)=0,\\
        \mu(\bigvee_{i \in I} a_i) = \sum_{i \in I} \mu(a_i), \mbox{ for any pairwise orthogonal sequence } (a_i)_{i \in I}.
    \end{gather}
    Then,
    \begin{enumerate}
        \item \(\mu\) is called a \emph{finitely additive measure} or a \emph{charge} if (1.2) holds for any finite set \(I\),
        \item \(\mu\) is called a \emph{signed measure} if (1.2) holds for any countably infinite set \(I\),
        \item \(\mu\) is called a \emph{completely additive signed measure} if (1.2) holds for any arbitrary set \(I\).
    \end{enumerate}
If \(\mu\) is nonnegative and \(\mu(1)=1\), then we call \(\mu\) a \emph{state}.
\end{definition}

\begin{example}
    Let \(H\) be a Hilbert space over a real or complex. The class
    \[L(H)=\{M \mid M \mbox{ is a closed subspace of } H\}\]
    is a quantum logic, where the partial ordering is the inclusion, the meet and join are defined as follows
    \[\bigwedge_{i \in I} M_i, \quad \bigvee_{i \in I} M_i = \mathrm{cl}(\mathrm{span(\bigcup_{i\in I} M_i)}),\]
    and the orthocomplementation \(\perp:M \mapsto M^\perp\).
\end{example}

\begin{example}
    Let \(H\) be a Hilbert space over a real or complex. The class
    \[\mathcal{P}(H)=\{P \mid P \mbox{ is a orthoprojector on } H\}\]
    is a quantum logic where \(P\le Q\) if and only if \[\left \langle Px,x \right \rangle \le \left \langle Qx,x \right \rangle\]
    for all \(x \in H\) and \(P^\perp=I-P\).
\end{example}

\bibliographystyle{abbrv}
\bibliography{dissertation}
\end{document}