\documentclass[a4paper,11pt]{report}
\usepackage{mathtools}
\usepackage{amssymb}
\usepackage{amsthm}
\theoremstyle{definition}
\newtheorem{definition}{Definition}
\newtheorem{example}{Example}
\newtheorem{exercise}{Exercise}

\theoremstyle{plain}
\newtheorem{theorem}{Theorem}
\newtheorem{lemma}{Lemma}
\newtheorem{proposition}{Proposition}
\newtheorem{corollary}{Corollary}

\theoremstyle{remark}
\newtheorem*{remark}{Remark}
\newtheorem*{claim}{Claim}

\title{Gleason's Theorem and Quantum Logic}
\author{Changjae Lee}

\begin{document}

\maketitle

\tableofcontents

\chapter*{Introduction}

In 1932, John von Neumann noticed that the projection operators of Hillbert space could be viewed as quantum mechanical propositions about observables \cite{Neumann2018}. The principles governing these quantum propositions were then called quantum logic by von Neumann and Birkhoff \cite{MR1503312}. Von Neumann also managed to dervie the Born rule in his textbook \cite{Neumann2018}. However, his assumptions were regarded to not be well-motivated by John Bell \cite{MR208927}, \cite{MR2726358}.

By the late 1940s, George Mackey was wondering whether the Born rule was the only possible rule for calculating probabilities in a theory that represented measurements as orthonormal bases on a Hilbert space. In his investigation of mathematical foundation of quantum mechanics, Mackey had proposed the following problem \cite[p.50-51]{MR96112}\cite[p.129]{MR1256736}: Determine all measures on the closed subspaces of a Hilbert space. Gleason provided the answer to this problem in the same year \cite{MR96113}, which is later called Gleason's theorem. Gleason succeeded in decribing all \(\sigma\)-additive probability measures on the logical of all closed subspaces of a separable Hilbert space and in showing that, except for the obvious two-dimensional counterexamplaes, all probability measures can be identified with normal states in the sense of von Neumann approach. Gleason's achievement confirmed von Neumann's original insight and put the calculus of Hilbert space quantum mechanics on natural physical grounds \cite[p.87-88]{MR2015280}. Gleason's theorem is of particular importance for the field of quantum logic and its attempt to find a minimal set of mathematical axioms for quantum theory.

After a considerable effort the Gleason's theorem was established in the early 90's for finitely additive vector measures on the projection lattices of von Neumann algebras. It has turned out that the lattice homomorphisms on nonabelian von Neumann algebras are \(\sigma\)-additve, or that finitely additive measures on projection lattices whose kernels are lattice ideal enjoy many continuity properties \cite[p.3-4]{MR2015280}.

The present paper is intended to serve as an introduction to topic of the relation between Gleason's theorem and the theory of quantum logic. Most of the contents were referred to chapters 1, 2, 3 and 5 of \cite{MR1256736} and chapters 1, 2, 3, and 9 of \cite{MR2015280}.


\chapter{Preliminaries}



\section{Elements of Operator Algebras}

Let \(A\) be a \(C^\ast\)-algebra. A \emph{state} in \(C^\ast\)-algebra is a positive functional \(\rho\) which is \(\left \lVert \rho \right \rVert=1\).


\begin{definition}
    A von Neumann algebra is a \(C^\ast\)-algebra that can be faithfully represented as a strongly operator closed \(\ast\)-algebra of \(B(H)\)
\end{definition}

\begin{theorem}[Bicommutant theorem]
    A \(\ast\)-subalgebra \(M\) of \(B(H)\) that contains the unit of \(B(H)\) if \(M\) is a
\end{theorem}



\chapter{Gleason's Theorem}

\section{Gleason's Theorem}

\begin{theorem}
    Let \(H\) be a Hilbert space with \(\dim H \ge 3\). Then any bounded completely additive measure \(\mu\) on the projection lattice \(P(H)\) extends uniquely to a normal functional on the algebra \(B(H)\) of all bounded operators action on \(H\).
\end{theorem}

\bibliographystyle{abbrv}
\bibliography{dissertation}
\end{document}