\chapter{Preliminaries}

\section{Elements of Operator Algebras}

A \(C^\ast\)-algebra \(A\) is a Banach algebra over the field \(\mathbb{C}\) together with a map \(\ast: A \to A\), \(a \mapsto a^\ast\) which following properties:
\begin{enumerate}
    \item \(a^{\ast\ast}=a\) for all \(a \in A\),
    \item \((a+b)^\ast=a^\ast+b^\ast\) for all \(a,b \in A\),
    \item \((ab)^\ast = b^\ast a^\ast\) for all \(a,b \in A\),
    \item \((\lambda a)^\ast = \bar{\lambda}a^\ast\) for all \(\lambda \in \mathbb{C}\) and \(a \in A\),
    \item \(\left \lVert aa^\ast \right \rVert = \left \lVert a \right \rVert \left \lVert a^\ast \right \rVert\) for all \(a \in A\).
\end{enumerate}

A bounded linear map \(\pi:A \to B\) between \(C^\ast\)-algebras \(A\) and \(B\) is called a \(\ast\)-homomorphism if
\begin{enumerate}
    \item \(\pi(ab)=\pi(a)\pi(b)\)
    \item \(\pi(a^\ast) = \pi(a)^\ast\)
\end{enumerate}
for all \(a,b \in A\).

Let \(A\) be a \(C^\ast\)-algebra. A \emph{state} in \(C^\ast\)-algebra is a positive functional \(\rho\) which is \(\left \lVert \rho \right \rVert=1\). We shall denote the set of all states on \(A\) by \(S(A)\) which is called the \emph{state space} of A.

A \(\ast\)-representation of a \(C^\ast\)-algebra \(A\) on a Hilbert space \(H\) is a ring homomorphism \(\pi: A \to B(H)=\{f \mid f \mbox{ is a bounded linear operator on }H\}\) such that
\begin{enumerate}
    \item \(\pi(a^\ast)=\pi(a)^\ast\) for all \(a \in A\),
    \item \(\pi\) is nondegenerate.
\end{enumerate}

\begin{theorem}[GNS representation]
    Given a state \(\rho\) of \(A\), there is a \(\ast\)-representation \(\pi\) of \(A\) acting on a Hilbert space \(H\) with unit cylic vector \(\psi\) such that
    \[\rho(a)=\left \langle \pi(a) \psi,\psi  \right \rangle\]
    for every \(a\) in \(A\). Moreover, the representation is unique up to unitary equivalence.
\end{theorem}
\begin{proof}
    \cite[Theorem 4.5.2]{MR1468229}
\end{proof}

Let \(X\) be an inner product space. A \emph{spilitting subspace} of \(X\) is the subspace \(M\) of \(X\) which is the following property holds,
\[M+M^\perp=S.\]
Then any vector \(x \in X\) can be uniquely expressed in the form
\[x=x_M+x_{M^{\perp}},\]
where \(x_M \in M\) and \(x_{M^\perp} \in M^\perp\). We denote by \(E(X)\) the set of all splitting subspaces.
The map \(P_M:X\to X\) such that \(P_M(x)=x_M\) for all \(x \in X\) is a bounded linear operator with \(\left \lVert P_M \right \rVert=1\) whenever \(M \neq \{0\}\). Moreover, \(P^2_M=P_M\), \(P_M\) is self-adjoint, and \(\mathrm{ran}\, P_M = M\). The operator \(P_M\) is called the \emph{orthoprojector} or \emph{projection} from \(X\) onto \(M\). It is clear that \(I-P_M\) is an orthoprojector onto \(M^\perp\).
\section{Quantum Logic}


