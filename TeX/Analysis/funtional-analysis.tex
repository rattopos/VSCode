\part{Functional Analysis}

Throughout this part \(\mathbb{F}\) will denote either \(\mathbb{C}\) or \(\mathbb{R}\)

\chapter{Topological Vector Spaces}

\chapter{Normed Vector Spaces}

\chapter{Banach Spaces}

\chapter{Hilbert Spaces}

\section{Inner Product Spaces}

Let \(\mathcal{X}\) be a \(\mathbb{F}\)-vector space. Consider some conditions on \(u:\mathcal{X}\times \mathcal{X} \to \mathbb{F}\);
\begin{enumerate}[label=(\alph*)]
    \item \(u(\alpha x + \beta y,z) = \alpha u(x,z) + \beta u(y,z)\)
    \item \(u(x,y)=\overline{u(y,x)}\)
    \item \(u(x,x) \ge 0\)
    \item \(u(x,x)=0 \Rightarrow x=0 \)
\end{enumerate}

If \(u\) satisfies (a)-(c), then we call \(u\) a \emph{semi-inner product} on \(\mathcal{X}\). An \emph{inner product} is a semi-inner product that also satisfies (d) and denote it by \(\left \langle \cdot,\cdot \right \rangle\).

\begin{theorem}[The Cauchy-Bunyakowsky-Schwart Inequality]
    If \(u\) is a semi-inner product on \(\mathcal{X}\), then
    \[\forall x,y \in \mathcal{X},\, \left \lvert u(x,y) \right \rvert^2 \le u(x,x) u(y,y)\]
    Moreover, equality occurs iff
    \[\exists \alpha,\beta \in \mathbb{F},\, u(\alpha x + \beta y,\alpha x + \beta y)=0\]
\end{theorem}

Let \(\mathcal{H}\) be a \(\mathbb{C}\)-vector space. If \(\mathcal{H}\) equipped with a semi-inner product, then \(\mathcal{H}\) is called \emph{semi-Hilbert space}. If the semi-inner product is an inner product, \(\mathcal{H}\) is called \emph{pre-Hilbert space}.

If \(\mathcal{H}\) is a pre-Hilbert space, we define
\[\left \lVert x \right \rVert = \sqrt{\left \langle x,x \right \rangle}\]