\documentclass[a4paper,12pt]{book}
\usepackage{amssymb,amsthm}
\usepackage{amsrefs}
\usepackage{mathtools}
\usepackage[scr=boondox,bb=stix]{mathalpha}
\usepackage{stmaryrd}
\usepackage{xparse}
\usepackage{hyperref}
\hypersetup{
    pdfpagemode=FullScreen,
    colorlinks=true,
    pdftitle={Summary of Analysis}
}
\usepackage{bookmark}
\usepackage{enumitem}

\usepackage{imakeidx}
\makeindex

\theoremstyle{definition}
\newtheorem{definition}{Definition}[section]
\newtheorem{example}{Example}[section]
\newtheorem{exercise}{Exercise}[section]

\theoremstyle{plain}
\newtheorem{theorem}{Theorem}
\newtheorem{lemma}{Lemma}
\newtheorem{proposition}{Proposition}
\newtheorem{corollary}{Corollary}

\theoremstyle{remark}
\newtheorem*{remark}{Remark}
\newtheorem*{claim}{Claim}

\DeclareEmphSequence{\bfseries}

\NewDocumentCommand{\Net}{O{x} O{\alpha} O{J}}{\left( #1_#2 \right)_{#2 \in #3}}
\NewDocumentCommand{\Seq}{O{f} O{n}}{\left( #1_#2 \right)_{#2 = 1}^{\infty}}

\title{Summary of Analysis}
\author{Changjae Lee}

\begin{document}

\frontmatter
\maketitle
\tableofcontents

\mainmatter
\part{Measure and Integration}

\chapter{Foundations}

\section{Sets}
Let \(A\) be a nonemptyset and \(R\) be a binary relation on \(A\).

\begin{description}
    \item[Refl] Reflexive: \(\forall a \in A,\, aRa\)
    \item[AntiRefl] Antireflexive: \(\forall a \in A,\, \neg aRa\)
    \item[Sym] Symmetric: \(\forall a,b \in A,\, aRb \Rightarrow bRa\)
    \item[AntiSym] Antisymmetic: \(\forall a,b \in A,\, aRb \mbox{ and } bRa \Rightarrow a=b\)
    \item[Asym] Asymmetric: \(\forall a,b \in A,\, aRb \Rightarrow \neg bRa\)
    \item[Trans] Transitive: \(\forall a,b \in A,\, aRb \mbox{ and } bRc \Rightarrow aRc\)
    \item[Total] Total: \(\forall a,b \in A,\, a \neq b \Rightarrow aRb \mbox{ or } bRa \)
    \item[Well-founded] \(\exists m \in A, \forall a \in A,\, mRa \)    
\end{description}

\begin{table}
\begin{centering}
    \begin{tabular}{ c l }
    \hline
    Name & Definition \\
    \hline
    equivalence & {\bf Refl + Sym + Trans} \\
    preorder & {\bf Refl + Trans} \\
    partial order & {\bf Refl + AntiSym + Trans} \\
    total order & {\bf Refl + AntiSym + Trans + Total} \\
    well-ordering & {\bf Refl + Sym + Trans + Total + Well-founded} \\
    \hline
    \end{tabular}
\end{centering}
    \caption{List of transitive relations.}
\end{table}

\begin{description}
    \item[proset] The pair \((A,\precsim) \) is called \emph{preordered set (proset)}\index{order!proset} if \(\precsim \) is a preorder on \(A\).
    \item[poset] The pair \((A,\preccurlyeq)\) is called \emph{partially ordered set (poset)}\index{order!poset} if \(\preceq\) is a partial order on \(A\).
    \item[toset] The pair \((A,\preceq)\) is called \emph{totally ordered set (toset)}\index{order!toset} if \(\preceq\) is a total order on \(A\).
    \item[woset] The pair \((A,\le)\) is called \emph{well-ordered set (woset)} if \(\le\)\index{order!woset} is a well-ordering on \(A\).
\end{description}

\section{Systems of Sets}

\cite{MR2977961}

\section{Nets and Filters}

\begin{definition}[directed set] Let \((J, \precsim)\) be a proset.
    \begin{description}
        \item[(upward) directed] \(\forall a,b \in J,\exists c \in J,\, a\precsim c \mbox{ and } b \precsim c \).
        \item[downward directed] \(\forall a,b \in J,\exists c \in J,\, c \precsim a \mbox{ and } c \precsim b\).
    \end{description}
\end{definition}

\begin{definition}
    A \emph{net}\index{net} \(x_\bullet = \Net \) in \(X\) is a function \(x_\bullet: J \to X\) where \(J\) is a directed set. 
\end{definition}


\section{Topological Properties}

\chapter{Measures}

\section{Abstract Measures}

If \(\Sigma\) is a \(\sigma\)-algebra on \(X\), then \((X,\Sigma)\) is called a \emph{measurable space}\index{measurable sapce}.

\begin{definition}
    A \emph{measure}\index{measure} on \((X,\Sigma)\) is a function \(\mu:\Sigma \to [0,\infty]\) s.t.
    \begin{enumerate}
        \item \(\mu(\varnothing)\) = 0
        \item (countable additivity) If \(\left( E_j \right)_{1}^{\infty} \) disjoint sequence in \(\Sigma\), then \[\mu(\bigcup_{j=1}^{\infty} E_j) = \sum_{j=1}^{\infty} \mu(E_j) \]
    \end{enumerate}
If \(\mu \) is a measure on \((X,\Sigma)\), then \((X,\Sigma,\mu)\) is called \emph{measure space}\index{measure space}.
\end{definition}

\begin{definition}
Let \((X,\Sigma,\mu)\) is a measuable space. Then \(\mu\) is
\begin{description}
    \item[finite] \(\mu(X)<\infty\)
    \item[\(\sigma\)-finite] \(\exists \Seq[E][j] \in \mathrm{Seq}(\Sigma),\, X = \bigcup_{j=1}^{\infty} E_j \mbox{ and } \mu(E_j)< \infty \)
    \item[semifinite] \(\forall E \in \Sigma,\exists F \in \Sigma,\, F \subset E \mbox{ and } 0<\mu(F)<\infty\)
\end{description}
\end{definition}

\paragraph{Iverson bracket}
    Let \(P\) be a statement,
    \[\left \llbracket P \right \rrbracket=\begin{dcases}
        1, & P \mbox{ is true}\\
        0, & \mbox{otherwise}
    \end{dcases}\]
Then the \emph{indicator function}\index{indicator function} of a subset \(A\) of a set \(X\) is a function \(\mathbb{1}_A : X \to \left\{ 0,1 \right\}\) defined as
\[\mathbb{1}_A(x) = \left \llbracket x \in A \right \rrbracket = \begin{dcases}
    1, & x \in A \\
    0, & \mbox{otherwise}
\end{dcases}\]
\begin{example}
    Let \(X\) be any nonempty set
    \begin{itemize}
        \item The \emph{counting measure}\index{measure!counting measure} \(\mu : \mathcal{P}(X) \to [0,\infty]\) is defined by
        \[\mu(E)=\begin{dcases}
            \left \lvert E \right \rvert, & E \mbox{ is finite} \\
            \infty, & \mbox{otherwise}
        \end{dcases}\]
        \item \(\forall (X,\Sigma), \forall x \in X\), a \emph{Dirac measure}\index{measure!Dirac measure} \(\delta_x: \Sigma \to [0,\infty]\) defined by \[\delta_x(E)=\mathbb{1}_E(x)=\left \llbracket x \in E \right \rrbracket \]
    \end{itemize}
\end{example}

\section{Borel Measures}

\section{Outer Measures}

\begin{definition}
    An \emph{outer measure}\index{measure!outer measure} on a nonempty set \(X\) is a function \(\mu^\ast: \mathcal{P}X \to [0,\infty]\) s.t.
    \begin{enumerate}
        \item \(\mu^\ast(\varnothing)=0\)
        \item \(A \subset B \Rightarrow \mu^\ast(A) \le \mu^\ast(B)\)
        \item \(\mu^\ast(\bigcup_{j=1}^{\infty} A_j) \le \sum_{1}^{\infty} \mu^\ast(A_j)\)
    \end{enumerate}
\end{definition}

\section{Complete Measures}


\chapter{Abstract Integration}

\section{Measurable Functions}

\section{Integration of Nonnegative Functions}

\section{Integration of Complex Functions}

\section{Product Measures}


\chapter{Signed Measure and Differentiation}

\section{Signed Measures}

\section{Complex Measures}

\section{Differentiation on Euclidean Space}

\section{Bounded Variations}


\part{Functional Analysis}

\chapter{Topological Vector Spaces}

\chapter{Banach Spaces}

\chapter{Hilbert Spaces}

\backmatter
\printindex

\end{document}