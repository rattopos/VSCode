\part{Lattices and Algebras}

\chapter{Lattice Theory}

\section{Lattices}

\begin{definition}
    A poset \(L\) is a \emph{lattice} if
    \[\forall a,b \in L,\, \sup\{a,b\} \in L \mbox{ and } \inf\{a,b\} \in L\]
\end{definition}

\begin{definition}
    Let \(A\) be a nonempty set. An \emph{n-ary operation} \(f\) on the \(A\) is a map \(A^n \to A\). We define \(A^0=\emptyset\).
    \begin{center}
        \begin{tabular}{|c|c|}
            \hline
            number & name \\
            \hline
            \(n=0\) & nullary operation \\
            \(n=1\) & unary opeartion \\
            \(n=2\) & binary operation \\
            \vdots & \vdots \\
            \hline
        \end{tabular}
    \end{center}
\end{definition}

\begin{definition}
    A \emph{universal algebra}, or simply \emph{algebra}, consists of a nonempty set \(A\) and a set \(F\) of operations; each \(f \in F\) is an \(n\)-ary operation for some \(n\) (depending on \(f\)). We denote this algebra by \(\mathfrak{A}\) or \((A;F)\).
\end{definition}

A \emph{type} \(\tau\) of algebras is a sequence \((n_0,n_1,\dots,n_\gamma,\dots)\) of nonnegative integers, \(\gamma<o(\tau)\), where \(o(\tau)\) is an ordinal called the \emph{order} of \(\tau\). An algebra \(\mathfrak{A} \) of type \(\tau\) is an ordered pari \((A;F)\), where \(A\) is a nonempty set and \(F\) is a sequence \((f_0,\dots,f_\gamma,\dots)\), where \(f_\gamma\) is an \(n_\gamma\)-ary operation on \(A\) for \(\gamma < o(\gamma)\).

\begin{definition}
    Let \(A;\circ\) be an alebra of type (2). If
    \begin{center}
        \begin{tabular}{r l l}
            (Idem) & Idempotent: & \(a \circ a = a\) \\
            (Comm) & Commutativity: & \(a \circ b = b \circ a\) \\
            (Assoc) & Associativity: & \((a\circ b) \circ c = a \circ (b \circ c)\)
        \end{tabular}
    \end{center}
    holds, then we call \((L,\circ)\) a \emph{semilattice}.
\end{definition}

\begin{definition}
    Let \((L,\wedge,\vee)\) be an algebra of type (2,2) is called a \emph{lattice} if \(L\) is a nonempty set, \((L;\vee)\) and \((L;\wedge)\) are semilattices, and
    \begin{center}
    \begin{tabular}{r l l}
        (Asorp) & Absorption: & \(\forall a,b \in L,\, a \vee (a\wedge b) = a\) and \(a\wedge (a\vee b) = a \).
    \end{tabular}
    \end{center}
\end{definition}

\begin{theorem}
    \begin{enumerate}[label=(\alph*)]
        \item Let the poset \(\mathfrak{L} =(L;\le)\) be a lattice. Set
        \begin{eqnarray*}
            a \vee b &=& \sup\{a,b\}, \\
            a \wedge b &=& \inf \{a,b\}.
        \end{eqnarray*}
        Then the algebra \(\mathfrak{L}^{\mathrm{alg}}=(L;\vee,\wedge)\) is a lattice.

        \item Let the algebra \(\mathfrak{L} = (L;\vee,\wedge) \) be a lattice. Set
        \begin{eqnarray*}
            a \vee b = b \Rightarrow a \le b
        \end{eqnarray*}
        Then \(\mathfrak{L}^{\mathrm{ord}}\) is a poset, and the \(\mathfrak{L}^{\mathrm{ord}}\) is a lattice.

        \item Let the poset \(\mathfrak{L}^{\mathrm{ord}}=(L;\le)\) be a lattice. Then \((\mathfrak{L}^{\mathrm{alg}})^{\mathrm{ord}}=\mathfrak{L}\).
        
        \item Let the algebra \(\mathfrak{L}=(L;\vee,\wedge)\) be a lattice. Then \((\mathfrak{L}^{\mathrm{ord}})^\mathrm{alg}=\mathfrak{L}\).
    \end{enumerate}

\section{Distributive Lattices}

\section{Modular Lattices}

\section{Complemented Lattices}

\section{Othomodular Lattices}

\chapter{Algebras}

\end{theorem}